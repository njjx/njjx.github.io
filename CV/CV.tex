\documentclass[letterpaper,11pt]{article}
\usepackage{pifont}

%-----------------------------------------------------------
\usepackage{latexsym}
\usepackage[empty]{fullpage}
\usepackage[usenames,dvipsnames]{color}
\usepackage{verbatim}
\usepackage{hyperref}
\usepackage{framed}
\usepackage{tocloft}
\usepackage{bibentry}
\usepackage{fancyhdr}
\pagestyle{fancy}
\usepackage{ragged2e}
\usepackage{multirow}
\hypersetup{
    colorlinks,%
    citecolor=blue,%
    filecolor=blue,%
    linkcolor=black,%
    urlcolor=black    % can put red here to better visualize the links
}
\urlstyle{same}
\definecolor{mygrey}{gray}{.85}
\definecolor{mygreylink}{gray}{.30}
\textheight=9.0in
\raggedbottom
\raggedright
\setlength{\tabcolsep}{0in}

% Adjust margins
\addtolength{\oddsidemargin}{-0.3in}
\addtolength{\evensidemargin}{0.3in}
\addtolength{\textwidth}{0.6in}
\addtolength{\topmargin}{-0.375in}
\addtolength{\textheight}{0.75in}

%-----------------------------------------------------------
%Custom commands

\newcommand{\resitem}[1]{\item #1 \vspace{-2pt}}
\newcommand{\resheading}[1]{{\large \colorbox{mygrey}{\begin{minipage}{\textwidth}{\textbf{#1 \vphantom{p\^{E}}}}\end{minipage}}}}

%-----------------------------------------------------------

%-----------------------------------------------------------
%General Resume Tips
%   No periods!  Technically, nothing in this document is a full sentence.
%   Use parallelism by ending key words with the same thing,  i.e. "Coordinated; Designed; Communicated".
%   More tips on bottom of this LaTeX document.
%-----------------------------------------------------------


\renewcommand{\rmdefault}{ptm}

\renewcommand\baselinestretch{1}% change line space here
\newcommand{\profchen}{Prof.~\href{https://www.medphysics.wisc.edu/blog/staff/chen-guanghong/} {Guang-Hong Chen}}

\newcommand{\profculberson}{Prof.~\href{https://www.medphysics.wisc.edu/blog/staff/culberson-wesley/} {Wesley~Culberson}}
\newcommand{\xji}{\textbf{X.~Ji}}
\begin{document}



\newcommand{\mywebheader}{
\begin{tabular*}{\textwidth}{l@{\extracolsep{\fill}}r}

	\end{tabular*}
\\
\vspace{0.35in}}

\lhead{\textbf{{Xu Ji}}}
\rhead{\href{mailto:xuji@seu.edu.cn}{xuji@seu.edu.cn}}
\cfoot{\thepage}
\renewcommand{\headwidth}{\textwidth}
%\renewcommand{\headrulewidth}{1pt}
%\renewcommand{\footrulewidth}{1pt}

% CHANGE HEADER SOURCE HERE
\mywebheader
\begin{tabular*}{\textwidth}{l @{\extracolsep{\fill}}r}
   \multirow{4}{*}{\textbf{\Huge Xu Ji}} &Tenure-track Associate Professor\\
  &School of Computer Science and Engineering, Southeast University\\
  &No.2 Sipailou Xuanwu District, Nanjing, 210096, China\\
  &\textbf{Email:} \href{mailto:xuji@seu.edu.cn}{xuji@seu.edu.cn} 
\end{tabular*}

\resheading{Education}
	\begin{itemize}
	\item \textbf{Ph.D. in Medical Physics}, University of Wisconsin-Madison\cftdotfill{\cftdotsep}09/2015 - 07/2020
  
  Thesis: Application of photon counting detectors to x-ray CT systems 
  
	Advisor: \profchen	
        \item
\textbf{B.S. in Physics}, Kuangyaming Honors School, Nanjing University\cftdotfill{\cftdotsep}09/2011 - 06/2015
		
GPA: 94/100~(3.9/4.0)
	
Ranking: 1/80 \qquad
       
\end{itemize} % End Education list
%%%%%%%%%%%%%%%%%%%%%%%
\resheading{Research Interests}
	\begin{itemize}
	\item Broad interests: X-ray imaging systems and algorithms
  \item Specific interests: I work on photon counting detector-based x-ray imaging and x-ray differential phase contrast imaging for medical applications. 
	\end{itemize}

%%%%%%%%%%%%%%%%%%%%%%
\resheading{Experiences}
\begin{itemize}
  \item \textbf{Tenure-track Associate Professor}\cftdotfill{\cftdotsep} 11/2021 - present

  School of Computer Science and Engineering at Southeast University
\item \textbf{Postdoctoral Researcher}\cftdotfill{\cftdotsep} 08/2020 - 07/2021

Department of Medical Physics at UW-Madison

\item \textbf{Graduate Research Assistant}\cftdotfill{\cftdotsep} 08/2015 - 07/2020

Department of Medical Physics at UW-Madison

Supervised by \profchen
\item \textbf{Graduate Teaching Assistant} \cftdotfill{\cftdotsep} Fall semesters in 2017,  2018 \& 2019

Course: Medical Physics 501 - Radiological Physics and Dosimetry

Instructor: \profchen~\& \profculberson
	 \item \textbf{Visiting Student}, Department of Physics, Duke University\cftdotfill{\cftdotsep}08/2014 - 12/2014
\end{itemize}
%%%%%%%%%%%%%%%%%%%%%%
\resheading{Publications}
\begin{itemize}
\item Journal publications
\begin{enumerate}\justifying
\item \xji, M. Feng, K. Treb, R. Zhang, S. Schafer and K. Li, “Development of an integrated C-arm interventional imaging system with a strip photon counting detector and a flat panel detector”, IEEE Trans. Med. Imag. (2021)
\item M. Feng*, \xji*, R. Zhang, K. Treb, A. M. Dinger and Ke Li, “An experimental method to correct low-frequency concentric artifacts in photon counting CT”, Phys. Med. Biol. 66 (17) 175011 (2021) (*co-first author)
\item \xji, K. Treb, and K. Li. “Anomalous edge response of cadmium telluride-based photon counting detectors jointly caused by high-flux radiation and inter-pixel communication,” Phys. Med. Biol., 66 (8) 085006 (2021)
\item \xji, R. Zhang, K. Li, and G.-H. Chen, ``Dual energy differential phase contrast CT (DE-DPC-CT) imaging,'' IEEE Trans. Med. Imag. (2020).
\item \xji, R. Zhang, K. Li, and G.-H. Chen, ``Is high sensitivity always desirable for a grating-based phase contrast imaging system?'' Med. Phys. 47: 1215-1228, (2019).
\item \xji, R. Zhang, G.-H. Chen, and K. Li, ``Task-driven optimization of the non-spectral mode of photon counting CT for intracranial hemorrhage assessment,'' Phys. Med. Biol. 64 215014 (2019).
\item E. Harvey, M. Feng, \xji, R. Zhang, Y. Li, G.-H. Chen, and K. Li, ``Impacts of photon counting CT to maximum intensity projection (MIP) images of cerebral CT angiography: theoretical and experimental studies,'' Phys. Med. Biol. 64 185015 (2019).
\item \xji, M. Feng, R. Zhang, G.-H. Chen, and K. Li, ``An experimental method to directly measure DQE(k) at k = 0 for 2D x-ray imaging systems,'' Phys. Med. Biol. 64 075013 (2019).
\item \xji, R. Zhang, G.-H. Chen, and K. Li, ``Impact of anti-charge sharing on the zerofrequency detective quantum efficiency of CdTe-based photon counting detector system: cascaded systems analysis and experimental validation,'' Phys. Med. Biol. 63, 095003 (2018).
\item Y. Ge*, \xji*, R. Zhang, K. Li, and G.-H. Chen, ``K-edge energy-based calibration method for photon counting detectors,'' Phys. Med. Biol. 63, 015022 (2018) (*co-first author)
\item \xji, Y. Ge, R. Zhang, K. Li, and G.-H. Chen, ``Studies of signal estimation bias in grating-based x-ray multicontrast imaging,'' Med. Phys. 44: 2453-2465, (2017).
\end{enumerate}

\item Conference proceedings
\begin{enumerate}\justifying
  \item K. Treb, \xji, R. Zhang, S. Schafer, and K. Li, “A hybrid photon counting and flat panel detector system for periprocedural hemorrhage monitoring in the angio suite,” Proc.~SPIE 115950U (2021).
\item \xji, R. Zhang, K. Li, and G.-H. Chen, ``Phase contrast CT enabled three-material decomposition in spectral CT imaging,'' Proc.~SPIE 113121B \& Oral presentation at SPIE Medical Imaging (2020).
\item M. Feng, \xji, R. Zhang, J. R. Miller, G.-H. Chen, K. Li, ``Impact of photon counting detector spectral distortion on virtual non-contrast CT imaging,'' Proc.~SPIE 113121J (2020).
\item \xji, R. Zhang, K. Li, and G.-H. Chen, ``Impact of the sensitivity factor on the signal-to-noise ratio in grating-based phase contrast imaging,'' Proc.~SPIE 10948 \& Oral presentation at SPIE Medical Imaging (2019).
\item \xji, M. Feng, R. Zhang, G.-H. Chen, and K. Li, ``An experimental method to correct drift-induced error in zero-frequency DQE measurement,'' Proc.~SPIE 10948 \& Oral presentation at SPIE Medical Imaging (2019).
\item M. Feng, \xji, K. Treb, R. Zhang, G.-H. Chen, K. Li, ``Spectrum optimization in photon counting detector based iodine K-edge CT imaging,'' Proc.~SPIE 10948 (2019).
\item E. Harvey, M. Feng, \xji, R. Zhang, G.-H. Chen, K. Li, ``Impacts of photon counting detector to cerebral CT angiography maximum intensity projection (MIP) images,'' Proc.~SPIE 10948 (2019).
\item \xji, R. Zhang, G.-H. Chen, and K. Li, ``Task-driven optimization of an experimental photon counting detector CT system for intracranial hemorrhage detection,'' Proc.~SPIE 10573 \& Oral presentation at SPIE Medical Imaging (2018).
\item K. Li, R. Zhang, J. Garrett, Y. Ge, \xji, and G.-H. Chen, ``Design, construction, and initial results of a prototype multi-contrast x-ray breast imaging system,'' Proc.~SPIE 10573 (2018).
\item \xji, R. Zhang, Y. Ge, K. Li, and G.-H. Chen, ``Signal and noise characteristics of a CdTe-based photon counting detector: cascaded systems analysis and experimental studies,'' Proc.~SPIE 10132 \& Oral presentation at SPIE Medical Imaging (2017).
\item \xji, Y. Ge, R. Zhang, K. Li, and G.-H. Chen, ``Weighted singular value decomposition (wSVD) to improve the radiation dose efficiency of grating-based x-ray phase contrast imaging with a photon counting detector,'' Proc.~SPIE 10132 \& Poster presentation at SPIE Medical Imaging (2017).
\item \xji, Y. Ge, R. Zhang, K. Li, and G.-H. Chen, ``Potential bias in signal estimation for grating-based x-ray multi-contrast imaging,'' Proc.~SPIE 10132 \& Oral presentation at SPIE Medical Imaging (2017).
\end{enumerate}
 
\item Conference abstracts 
\begin{enumerate}\justifying
\item \xji, S. Periyasamy, M. Feng, G.-H. Chen, P. F. Laeseke, and Ke Li, ``Development of a prototype c-arm photon counting detector CT system for Interventional Imaging: First In Vivo Animal Results,'' Oral presentation at AAPM (2021).

\item \xji, M. Feng, R. Zhang, G.-H. Chen, and K. Li, ``An experimental method to measure zero-Frequency DQE in the presence of system drift,'' Oral presentation at AAPM (2019).

\item \xji, M. Feng, R. Zhang, G.-H. Chen, and K. Li, ``A practical model for the energy response function of photon counting detector systems with anti-charge sharing logic,'' Oral presentation at AAPM (2019).

\item \xji, R. Zhang, G.-H. Chen, and K. Li, ``How does anti-charge sharing impact the zero-frequency DQE of photon counting detector systems? Theoretical framework and experimental validation,'' Oral presentation at AAPM (2018).

\item \xji, Y. Ge, R. Zhang, G.-H. Chen and K. Li, ``Potential application of photon counting detector CT in intracranial hemorrhage detection,'' Oral presentation at RSNA (2017).

\item Y. Ge, R. Zhang, J. W. Garrett, \xji, J. P. Cruz-Bastida, G.-H. Chen and K. Li, ``Initial experimental results from the first x-Ray dark field breast tomosynthesis prototype system,'' RSNA (2017).

\item \xji, Y. Ge, R. Zhang, K. Li and G.-H. Chen , ``Is a high sensitivity interferometer always good for a grating-based differential phase contrast imaging system?'' Oral presentation at XNPIG (2017).

\item Y. Ge, \xji, R. Zhang, K. Li, and G.-H. Chen, ``Energy calibration of photon counting detectors based on measurement of x-ray attenuation curve of K-edge materials,'' AAPM (2017).

\item Y. Ge, \xji, R. Zhang, K. Li, and G.-H. Chen, ``Radiation dose reduction in x-ray differential phase contrast breast imaging using an energy-resolved grating interferometer,'' RSNA (2016).

\item \xji, Y. Ge, R. Zhang, K. Li, and G.-H. Chen, ``Low dose performance of a CdTe single photon counting detector and its application in radiation dose reduction for x-ray differential phase contrast imaging,'' Oral presentation at RSNA (2016).
\end{enumerate}
\end{itemize}

%%%%%%%%%%%%%%%%%%%%%%
\resheading {Invited Talks}
\begin{enumerate}\justifying
\item ``Statistical properties of grating-based x-ray phase contrast imaging,'' Presented at Shenzhen Institutes of Advanced Technology, Chinese Academy of Science (2019).
\end{enumerate}
%%%%%%%%%%%%%%%%%%%%%%
\resheading{{Honors and Awards}}
\begin{itemize}\justifying
\item Runner-up, Robert F. Wagner all-conference best student paper award, SPIE Medical Imaging~(2020).
\item 1st place, Physics of medical imaging student paper award, SPIE Medical Imaging (2020).
\item Trainee research prize, RSNA (2017).
\item Expanding horizons grant award, AAPM (2016).
\item Student travel award, RSNA (2016).
\item China National Scholarship (2012).
\end{itemize}
%%%%%%%%%%%%%%%%%%%%%%%%%%%%%%%%
\resheading{Services}
\begin{itemize}
\item Reviewer of IEEE Transactions of Medical Imaging, Medical Physics, Journal of Applied Clinical Medical Physics, etc. 
\end{itemize}
%%%%%%%%%%%%%%%%%%%%%%
\resheading{Memberships}
\begin{itemize}
\item AAPM student member \cftdotfill{\cftdotsep} 2016 - 2020
\end{itemize}
%%%%%%%%%%%%%%%%%%%%%%
\resheading{Professional Certifications}
\begin{itemize}
\item The American Board of Radiology - Medical Physics - Part 1
\end{itemize}
\end{document}
