\documentclass[letterpaper,11pt]{article}
%\usepackage{pifont}
\usepackage[UTF8]{ctex}
%-----------------------------------------------------------
\usepackage{latexsym}
\usepackage[empty]{fullpage}
\usepackage[usenames,dvipsnames]{color}
\usepackage{verbatim}
\usepackage{hyperref}
\usepackage{framed}
\usepackage{tocloft}
\usepackage{bibentry}
\usepackage{fancyhdr}
\pagestyle{fancy}
\usepackage{ragged2e}
\usepackage{multirow}
\hypersetup{
    colorlinks,%
    citecolor=blue,%
    filecolor=blue,%
    linkcolor=black,%
    urlcolor=black    % can put red here to better visualize the links
}
\urlstyle{same}
\definecolor{mygrey}{gray}{.85}
\definecolor{mygreylink}{gray}{.30}
\textheight=9.0in
\raggedbottom
\raggedright
\setlength{\tabcolsep}{0in}

% Adjust margins
\addtolength{\oddsidemargin}{-0.0in}
\addtolength{\evensidemargin}{-0.0in}
\addtolength{\textwidth}{0.0in}
\addtolength{\topmargin}{-0.375in}
\addtolength{\textheight}{0.75in}

%-----------------------------------------------------------
%Custom commands

\newcommand{\resitem}[1]{\item #1 \vspace{-2pt}}
\newcommand{\resheading}[1]{{\large \colorbox{mygrey}{\begin{minipage}{\textwidth}{\textbf{#1 \vphantom{p\^{E}}}}\end{minipage}}}}



\renewcommand{\rmdefault}{ptm}

\renewcommand\baselinestretch{1}% change line space here
\newcommand{\profchen}{\href{https://www.medphysics.wisc.edu/blog/staff/chen-guanghong/} {Guang-Hong Chen 教授}}

\newcommand{\profculberson}{\href{https://www.medphysics.wisc.edu/blog/staff/culberson-wesley/} {Wesley~Culberson 教授}}
\newcommand{\xji}{\textbf{X.~Ji}}
\begin{document}
\begin{CJK}{UTF8}{song}


\newcommand{\mywebheader}{
\begin{tabular*}{\textwidth}{l@{\extracolsep{\fill}}r}

	\end{tabular*}
\\
\vspace{0.35in}}

\lhead{{{季续}}}
\rhead{\href{mailto:xji32@wisc.edu}{xji32@wisc.edu}}
\cfoot{\thepage}
\renewcommand{\headwidth}{\textwidth}
%\renewcommand{\headrulewidth}{1pt}
%\renewcommand{\footrulewidth}{1pt}

% CHANGE HEADER SOURCE HERE
\mywebheader
\begin{tabular*}{\textwidth}{l @{\extracolsep{\fill}}r}
   \multirow{4}{*}{\textbf{\Huge 季续}} &博士后\\
  &美国威斯康辛麦迪逊大学医学物理系\\
  &地址:1111 Highland Ave, Madison, WI 53705, US\\
  &{电子邮箱:} \href{mailto:xji32@wisc.edu}{xji32@wisc.edu} 
\end{tabular*}

\resheading{教育背景}
	\begin{itemize}
	\item 博士,美国威斯康辛大学麦迪逊分校医学物理专业\cftdotfill{\cftdotsep}08/2015 - 07/2020
  
  博士论文:Application of photon counting detectors to x-ray CT systems

	导师:\profchen	
        \item
学士,南京大学匡亚明学院物理专业\cftdotfill{\cftdotsep}09/2011 - 06/2015
		
学分绩:94/100~(3.9/4.0)
	
排名:1/80 \qquad
       
\end{itemize} % End Education list
%%%%%%%%%%%%%%%%%%%%%%%
\resheading{研究方向}
	\begin{itemize}\justifying
	\item 主要研究方向为x光影像系统及算法
	\item 详细研究方向为基于光子计数探测器的x光医学影像系统以及x光相衬成像在医学领域的应用
	\end{itemize}
%%%%%%%%%%%%%%%%%%%%%%
\resheading{工作经历}
\begin{itemize}
  \item \textbf{博士后}\cftdotfill{\cftdotsep} 08/2020 - 今

美国威斯康辛大学麦迪逊分校医学物理系
\item \textbf{研究助理}\cftdotfill{\cftdotsep} 08/2015 - 07/2020

美国威斯康辛大学麦迪逊分校医学物理系

导师:\profchen
\item \textbf{教学助理} \cftdotfill{\cftdotsep} 2017,2018及2019秋季学期

课程: 放射物理与辐射剂量学

主讲教师: \profchen与\profculberson
	 \item \textbf{访问学生}, 美国杜克大学物理系\cftdotfill{\cftdotsep}08/2014 - 12/2014
\end{itemize}
%%%%%%%%%%%%%%%%%%%%%%
\resheading{论文发表}
\begin{itemize}
\item 期刊论文
\begin{enumerate}\justifying
\item \xji, R. Zhang, K. Li, and G.-H. Chen, ``Dual energy differential phase contrast CT (DE-DPC-CT) imaging,'' IEEE Trans. Med. Imag. (2020).
\item \xji, R. Zhang, K. Li, and G.-H. Chen, ``Is high sensitivity always desirable for a grating-based phase contrast imaging system?'' Med. Phys. 47: 1215-1228, (2019).
\item \xji, R. Zhang, G.-H. Chen, and K. Li, ``Task-driven optimization of the non-spectral mode of photon counting CT for intracranial hemorrhage assessment,'' Phys. Med. Biol. 64 215014 (2019).
\item E. Harvey, M. Feng, \xji, R. Zhang, Y. Li, G.-H. Chen, and K. Li, ``Impacts of photon counting CT to maximum intensity projection (MIP) images of cerebral CT angiography: theoretical and experimental studies,'' Phys. Med. Biol. 64 185015 (2019).
\item \xji, M. Feng, R. Zhang, G.-H. Chen, and K. Li, ``An experimental method to directly measure DQE(k) at k = 0 for 2D x-ray imaging systems,'' Phys. Med. Biol. 64 075013 (2019).
\item \xji, R. Zhang, G.-H. Chen, and K. Li, ``Impact of anti-charge sharing on the zerofrequency detective quantum efficiency of CdTe-based photon counting detector system: cascaded systems analysis and experimental validation,'' Phys. Med. Biol. 63, 095003 (2018).
\item Y. Ge*, \xji*, R. Zhang, K. Li, and G.-H. Chen, ``K-edge energy-based calibration method for photon counting detectors,'' Phys. Med. Biol. 63, 015022 (2018) (*co-first author)
\item \xji, Y. Ge, R. Zhang, K. Li, and G.-H. Chen, ``Studies of signal estimation bias in grating-based x-ray multicontrast imaging,'' Med. Phys. 44: 2453-2465, (2017).
\end{enumerate}

\item 会议论文
\begin{enumerate}\justifying
\item \xji, R. Zhang, K. Li, and G.-H. Chen, ``Phase contrast CT enabled three-material decomposition in spectral CT imaging,'' Proc.~SPIE 113121B \& Oral presentation at SPIE Medical Imaging (2020).
\item M. Feng, \xji, R. Zhang, J. R. Miller, G.-H. Chen, K. Li, ``Impact of photon counting detector spectral distortion on virtual non-contrast CT imaging,'' Proc.~SPIE 113121J (2020).
\item \xji, R. Zhang, K. Li, and G.-H. Chen, ``Impact of the sensitivity factor on the signal-to-noise ratio in grating-based phase contrast imaging,'' Proc.~SPIE 10948 \& Oral presentation at SPIE Medical Imaging (2019).
\item \xji, M. Feng, R. Zhang, G.-H. Chen, and K. Li, ``An experimental method to correct drift-induced error in zero-frequency DQE measurement,'' Proc.~SPIE 10948 \& Oral presentation at SPIE Medical Imaging (2019).
\item M. Feng, \xji, K. Treb, R. Zhang, G.-H. Chen, K. Li, ``Spectrum optimization in photon counting detector based iodine K-edge CT imaging,'' Proc.~SPIE 10948 (2019).
\item E. Harvey, M. Feng, \xji, R. Zhang, G.-H. Chen, K. Li, ``Impacts of photon counting detector to cerebral CT angiography maximum intensity projection (MIP) images,'' Proc.~SPIE 10948 (2019).
\item \xji, R. Zhang, G.-H. Chen, and K. Li, ``Task-driven optimization of an experimental photon counting detector CT system for intracranial hemorrhage detection,'' Proc.~SPIE 10573 \& Oral presentation at SPIE Medical Imaging (2018).
\item K. Li, R. Zhang, J. Garrett, Y. Ge, \xji, and G.-H. Chen, ``Design, construction, and initial results of a prototype multi-contrast x-ray breast imaging system,'' Proc.~SPIE 10573 (2018).
\item \xji, R. Zhang, Y. Ge, K. Li, and G.-H. Chen, ``Signal and noise characteristics of a CdTe-based photon counting detector: cascaded systems analysis and experimental studies,'' Proc.~SPIE 10132 \& Oral presentation at SPIE Medical Imaging (2017).
\item \xji, Y. Ge, R. Zhang, K. Li, and G.-H. Chen, ``Weighted singular value decomposition (wSVD) to improve the radiation dose efficiency of grating-based x-ray phase contrast imaging with a photon counting detector,'' Proc.~SPIE 10132 \& Poster presentation at SPIE Medical Imaging (2017).
\item \xji, Y. Ge, R. Zhang, K. Li, and G.-H. Chen, ``Potential bias in signal estimation for grating-based x-ray multi-contrast imaging,'' Proc.~SPIE 10132 \& Oral presentation at SPIE Medical Imaging (2017).
\end{enumerate}
 
\item 会议摘要
\begin{enumerate}\justifying
\item \xji, M. Feng, R. Zhang, G.-H. Chen, and K. Li, ``An experimental method to measure zero-Frequency DQE in the presence of system drift,'' Oral presentation at AAPM (2019).

\item \xji, M. Feng, R. Zhang, G.-H. Chen, and K. Li, ``A practical model for the energy response function of photon counting detector systems with anti-charge sharing logic,'' Oral presentation at AAPM (2019).

\item \xji, R. Zhang, G.-H. Chen, and K. Li, ``How does anti-charge sharing impact the zero-frequency DQE of photon counting detector systems? Theoretical framework and experimental validation,'' Oral presentation at AAPM (2018).

\item \xji, Y. Ge, R. Zhang, G.-H. Chen and K. Li, ``Potential application of photon counting detector CT in intracranial hemorrhage detection,'' Oral presentation at RSNA (2017).

\item Y. Ge, R. Zhang, J. W. Garrett, \xji, J. P. Cruz-Bastida, G.-H. Chen and K. Li, ``Initial experimental results from the first x-Ray dark field breast tomosynthesis prototype system,'' RSNA (2017).

\item \xji, Y. Ge, R. Zhang, K. Li and G.-H. Chen , ``Is a high sensitivity interferometer always good for a grating-based differential phase contrast imaging system?'' Oral presentation at XNPIG (2017).

\item Y. Ge, \xji, R. Zhang, K. Li, and G.-H. Chen, ``Energy calibration of photon counting detectors based on measurement of x-ray attenuation curve of K-edge materials,'' AAPM (2017).

\item Y. Ge, \xji, R. Zhang, K. Li, and G.-H. Chen, ``Radiation dose reduction in x-ray differential phase contrast breast imaging using an energy-resolved grating interferometer,'' RSNA (2016).

\item \xji, Y. Ge, R. Zhang, K. Li, and G.-H. Chen, ``Low dose performance of a CdTe single photon counting detector and its application in radiation dose reduction for x-ray differential phase contrast imaging,'' Oral presentation at RSNA (2016).
\end{enumerate}
\end{itemize}

%%%%%%%%%%%%%%%%%%%%%%
\resheading {受邀报告}
\begin{enumerate}\justifying
\item ``Statistical properties of grating-based x-ray phase contrast imaging,'' 中科院深圳先进技术研究院 (2019).
\end{enumerate}
%%%%%%%%%%%%%%%%%%%%%%
\resheading{主要奖项}
\begin{itemize}\justifying
\item 国际光电工程学会医学影像会议 Robert F. Wagner 大会最佳学生论文奖第二名~(2020).
\item 国际光电工程学会医学影像会议~物理分会学生最佳论文奖第一名~(2020).
\item 北美放射学会~Trainee research prize~(2017).
\item 美国医学物理协会~Expanding horizons grant award~(2016).
\item 北美放射学会~Student travel award~(2016).
\item 中国国家奖学金~(2012).
\end{itemize}
%%%%%%%%%%%%%%%%%%%%%%%%%%%%%%%%
\resheading{学术服务}
\begin{itemize}
\item 《Medical Physics》《Journal of Applied Clinical Medical Physics》《The International Journal of Cardiovascular Imaging》 等期刊审稿人
\end{itemize}
%%%%%%%%%%%%%%%%%%%%%%
\resheading{会员}
\begin{itemize}
\item 美国医学物理协会学生会员 \cftdotfill{\cftdotsep} 2016 - 2020
\end{itemize}
%%%%%%%%%%%%%%%%%%%%%%
\resheading{专业证书}
\begin{itemize}
\item The American Board of Radiology - Medical Physics - Part 1
\end{itemize}
\end{CJK}
\end{document}
